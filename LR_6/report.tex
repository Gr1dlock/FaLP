\usepackage{cmap}					% поиск в PDF
\usepackage{mathtext} 				% русские буквы в формулах
\usepackage[T2A]{fontenc}			% кодировка
\usepackage[utf8]{inputenc}		% кодировка исходного текста
\usepackage[english,russian]{babel}	% локализация и переносы
\usepackage{indentfirst}
\frenchspacing

\usepackage[
bookmarks=true, colorlinks=true, unicode=true,
urlcolor=blue,linkcolor=black, anchorcolor=black,
citecolor=black, menucolor=black, filecolor=black,
]{hyperref}

\usepackage{caption}
\usepackage{csquotes}
\usepackage{multicol}

\usepackage{amsmath,amsfonts,amssymb,amsthm,mathtools} 
\usepackage{wasysym}
\usepackage{icomma}
\usepackage{wrapfig}

\usepackage{graphicx}

\usepackage{listings} 

\usepackage{geometry}
\geometry{left=2cm}
\geometry{right=1.5cm}
\geometry{top=2cm}
\geometry{bottom=2cm}


\usepackage{array,tabularx,tabulary,booktabs} % Дополнительная работа с таблицами
\usepackage{longtable}  % Длинные таблицы
\usepackage{multirow} % Слияние строк в таблице

\usepackage{soul}

\usepackage{xcolor}
\DeclareCaptionFont{white}{\color{white}}
\DeclareCaptionFormat{listing}{\colorbox{gray}{\parbox{\textwidth}{#1#2#3}}}
\captionsetup[lstlisting]{format=listing,labelfont=white,textfont=white}


\lstset{ 
	language=Prolog,                 % выбор языка для подсветки
	basicstyle=\small\ttfamily,
	numbers=left,               % где поставить нумерацию строк (слева\справа)
	numberstyle=\tiny\color{black},           % размер шрифта для номеров строк
	stepnumber=1,                   % размер шага между двумя номерами строк
	numbersep=5pt,                % как далеко отстоят номера строк от подсвечиваемого кода
	showspaces=false,
	backgroundcolor=\color{white},         
	showstringspaces=false,      % показывать или нет пробелы в строках
	showtabs=false,             % показывать или нет табуляцию в строках
	frame=single,              % рисовать рамку вокруг кода
	rulecolor = \color{black},
	tabsize=2,                 % размер табуляции по умолчанию равен 2 пробелам
	captionpos=t,              % позиция заголовка вверху [t] или внизу [b] 
	breaklines=true,           % автоматически переносить строки (да\нет)
	breakatwhitespace=true, % переносить строки только если есть пробел
	escapeinside={\%*}{*)},
}





\begin{document}
	
\begin{figure}[h!]
	\begin{center}
		{\includegraphics[width = \textwidth]{titul.png}}
	\end{center}
\end{figure}

\vspace*{20mm}

\huge
\begin{center}
	Лабораторная работа №6
\end{center}


\vspace*{50mm}

\large
\begin{flushleft}
	Студент: Луговой Д.М. \\
	Группа: ИУ7-61Б \\
	Преподаватель: Толпинская Н.Б.
\end{flushleft}

\vspace*{60mm}

\large
\begin{center}
	Москва, 2020 г.
\end{center}

\thispagestyle{empty}

\newpage
\vspace*{10mm}
\textbf{Цель работы}: приобрести навыки работы в Common Lisp.\\

\textbf{Задачи работы}: изучить работу интерпретатора Lisp, алгоритм работы функции eval, структуру и порядок обработки программы в Lisp.

\begin{enumerate}
\item \textbf{Способы определения функций}

Новые функции можно определить с помощью оператора defun. Он принимает три или более аргументов: имя, список параметров и ноль или более выражений, которые составляют тело функции. 

(defun func\_name (arg1 arg2 ... argN) func\_body)

Но функция не обязательно должна иметь имя, для того, чтобы определить функцию, не имеющую имени, необходимо воспользоваться лямбда-выражением. Лямбда-выражение – это список, содержащий символ lambda и следующие за ним список аргументов и тело, состоящее из нуля или более выражений.

(lambda (arg1 arg2 .. argN) func\_body)

\item \textbf{Вызовы функций, блокировка их выполнения}

Для простейшего вызова функции используется список, первый элемент которого трактуется как имя функции, а остальные - как ее аргументы. Любое введенное S-выражение передается функции EVAL, которая вычисляет его и возвращает результат его вычисления. Схема работы функции EVAL:
\begin{figure}[ht!]
\center{\includegraphics[scale=0.82]{FaLP1.png}}
\end{figure}

\newpage
\vspace*{10mm}
Для явного вызова функции в Lisp используются функционалы APPLY и FUNCALL. APPLY принимает первым аргументом функцию, а вторым - список аргументов, которые будут переданы в функцию. FUNCALL принимает переменное число аргументов, первый из которых функция, а остальные - аргументы, которые будут переданы в функцию. Оба функционала возвращают результаты вычисления переданной им функции при переденной ей аргументах.

Для того, чтобы заблокировать вычисление S-выражения используется функция QUOTE, или ее обозначение '. Она принимает S-выражение и возвращает его же, таким образом блокируя его вычисление.

\item \textbf{Локальное и глобальное определение значения атома}

Локальные значение атома можно определить с помощью функций let и let*. Областью видимости является тело функции, в которой определена переменная.

Синтаксис:

(let ((var1 value1) (var2 value2) ... (varN valueN)) body)

Сначала вычисляются значения value1, value2, ... , valueN, а затем происходит их связывание с var1, var2, ... , varN. 

(let* ((var1 value1) (var2 value2) ... (varN valueN)) body)

Отличие от let состоит в том, что связывание каждого значения value с символом var происходит сразу после вычисления значения.

Глобальные значение атома устанавливается с помощью функции setf. Областью видимости является весь код, следующий после определения.

Синтаксис: 

(setf var value)
\end{enumerate}
\newpage
\vspace*{10mm}
{\LARGE Задание №1}\\

Переписать функцию how-alike, приведенную в лекции и использующую COND, использую конструкции IF, AND/OR.

\begin{lstlisting}[caption=Переписанная функция how-alike]
(defun how-alike (x y) (if (or (= x y)(equal x y)) 'the_same
	(if (and (oddp x)(oddp y)) 'both_odd
		(if (and (evenp x)(evenp y)) 'both_even
			 'difference
			)
		)
	)
)
\end{lstlisting}

Пример:
\begin{lstlisting}
> (how_alike 1 1)
THE_SAME
> (how_alike 1 6)
DIFFERENCE
> (how_alike 8 4)
BOTH_EVEN
> (how_alike 5 7)
BOTH_ODD
\end{lstlisting}

{\LARGE Задание №2}\\

Даны 2 списка: с названиями стран и с названиями столиц.

Написать функции для создания:
\begin{itemize}
\item списка из двухэлементых списков, содержащих страну и столицу;
\item списка из точечных пар, содержащих страну и столицу.
\end{itemize}

\begin{lstlisting}[caption=Функция создания списка из двухэлементных списков]
(defun join_in_lists (countries capitals) (mapcar #'(
	lambda 
		(country capital)
		(cons country 
			(cons capital nil)
		)
	) 
	countries capitals
))
\end{lstlisting}

\newpage
\vspace*{10mm}

Пример:
\begin{lstlisting}
> (join_in_lists '(Russia England USA Ukraine) '(Moscow London Washington Kiev))
((RUSSIA MOSCOW) (ENGLAND LONDON) (USA WASHINGTON) (UKRAINE KIEV))
\end{lstlisting}

\begin{lstlisting}[caption=Функция создания списка из точечных пар]
(defun join_in_pairs (countries capitals)(mapcar #'cons countries capitals))
\end{lstlisting}

Пример:
\begin{lstlisting}
> (join_in_pairs '(Russia England USA Ukraine) '(Moscow London Washington Kiev))
((RUSSIA . MOSCOW) (ENGLAND . LONDON) (USA . WASHINGTON) (UKRAINE . KIEV))
\end{lstlisting}

Написать функции для поиска страны по столице и столицы по стране.

\begin{lstlisting}[caption=Функция поиска в списке двухэлементных списков]
(defun find_in_lists_list (name lst) (cond 
	((OR
		(car 
			(rassoc (cons name nil) lst :test 'equal)
		)
		(cadr
			(assoc name lst)
		)
	))
))
\end{lstlisting}
Пример:

\begin{lstlisting}
> (setf lists (join_in_lists '(Russia England USA Ukraine) '(Moscow London Washington Kiev))
((RUSSIA MOSCOW) (ENGLAND LONDON) (USA WASHINGTON) (UKRAINE KIEV))
> (find_in_lists_list 'London lists)
ENGLAND
> (find_in_lists_list 'USA lists)
WASHINGTON
> (find_in_lists_list 'Paris lists)
NIL
\end{lstlisting}

\newpage
\vspace*{10mm}

\begin{lstlisting}[caption=Функция поиска в списке точечных пар]
(defun find_in_pairs_list (name lst) (cond 
	((OR
		(car 
			(rassoc name lst)
		)
		(cdr
			(assoc name lst)
		)
	))
))
\end{lstlisting}

Пример:
\begin{lstlisting}
> (setf pairs (join_in_pairs '(Russia England USA Ukraine) '(Moscow London Washington Kiev)))
((RUSSIA . MOSCOW) (ENGLAND . LONDON) (USA . WASHINGTON) (UKRAINE . KIEV))
> (find_in_pairs_list 'Kiev pairs)
UKRAINE
> (find_in_pairs_list 'Russia pairs)
MOSCOW
> (find_in_pairs_list 'India pairs)
NIL
\end{lstlisting}

\end{document}