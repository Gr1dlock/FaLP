\usepackage{cmap}					% поиск в PDF
\usepackage{mathtext} 				% русские буквы в формулах
\usepackage[T2A]{fontenc}			% кодировка
\usepackage[utf8]{inputenc}		% кодировка исходного текста
\usepackage[english,russian]{babel}	% локализация и переносы
\usepackage{indentfirst}
\frenchspacing

\usepackage[
bookmarks=true, colorlinks=true, unicode=true,
urlcolor=blue,linkcolor=black, anchorcolor=black,
citecolor=black, menucolor=black, filecolor=black,
]{hyperref}

\usepackage{caption}
\usepackage{csquotes}
\usepackage{multicol}

\usepackage{amsmath,amsfonts,amssymb,amsthm,mathtools} 
\usepackage{wasysym}
\usepackage{icomma}
\usepackage{wrapfig}

\usepackage{graphicx}

\usepackage{listings} 

\usepackage{geometry}
\geometry{left=2cm}
\geometry{right=1.5cm}
\geometry{top=2cm}
\geometry{bottom=2cm}


\usepackage{array,tabularx,tabulary,booktabs} % Дополнительная работа с таблицами
\usepackage{longtable}  % Длинные таблицы
\usepackage{multirow} % Слияние строк в таблице

\usepackage{soul}

\usepackage{xcolor}
\DeclareCaptionFont{white}{\color{white}}
\DeclareCaptionFormat{listing}{\colorbox{gray}{\parbox{\textwidth}{#1#2#3}}}
\captionsetup[lstlisting]{format=listing,labelfont=white,textfont=white}


\lstset{ 
	language=Prolog,                 % выбор языка для подсветки
	basicstyle=\small\ttfamily,
	numbers=left,               % где поставить нумерацию строк (слева\справа)
	numberstyle=\tiny\color{black},           % размер шрифта для номеров строк
	stepnumber=1,                   % размер шага между двумя номерами строк
	numbersep=5pt,                % как далеко отстоят номера строк от подсвечиваемого кода
	showspaces=false,
	backgroundcolor=\color{white},         
	showstringspaces=false,      % показывать или нет пробелы в строках
	showtabs=false,             % показывать или нет табуляцию в строках
	frame=single,              % рисовать рамку вокруг кода
	rulecolor = \color{black},
	tabsize=2,                 % размер табуляции по умолчанию равен 2 пробелам
	captionpos=t,              % позиция заголовка вверху [t] или внизу [b] 
	breaklines=true,           % автоматически переносить строки (да\нет)
	breakatwhitespace=true, % переносить строки только если есть пробел
	escapeinside={\%*}{*)},
}





\begin{document}
	
\begin{figure}[h!]
	\begin{center}
		{\includegraphics[width = \textwidth]{titul.png}}
	\end{center}
\end{figure}

\vspace*{20mm}

\huge
\begin{center}
	Лабораторная работа №8\\
	Использование функционалов
\end{center}


\vspace*{35mm}

\large
\begin{flushleft}
	Студент: Луговой Д.М. \\
	Группа: ИУ7-61Б \\
	Преподаватель: Толпинская Н.Б.
\end{flushleft}

\vspace*{55mm}

\large
\begin{center}
	Москва, 2020 г.
\end{center}

\thispagestyle{empty}

\textbf{Цель работы}: приобрести навыки работы в управляющими структурами Lisp.\\

\textbf{Задачи работы}: изучить работу функций с произвольным количеством аргументов, функций, разрушающих и не разрушающих структуру исходных аргументов.\\

\section*{Лабораторная работа 5}

\subsection*{Задание 1}

Написать функцию, которая по своему списку-аргументу lst определяет, является ли он палиндромом (то есть равны ли lst и (reverse lst)).

\begin{lstlisting}[caption=1-я реализация функции проверки на палиндром]
(defun check_pal(lst)
	(cond
		(
			(null lst)
		)
		(
			(and 
				(equal 
					(car lst) 
					(mapcon #'(lambda (el) 
							(cond
								(
									(null (cdr el))
									(car el)
								)
							)
						) lst
					)
				)
				(check_pal 
					(mapcon #'(lambda (el) 
							(cond 
								(
									(cdr el)
									(cons (car el) nil)
								)
							)
						) (cdr lst)
					)
				)
			)
		)
	)
)
\end{lstlisting}
\textbf{lst} - входной список.

Условием выхода из рекурсии является достижение середины списка(аргумент - nil) - возвращается t. 

Осуществляется проверка на равенство первого и последнего элементов списка, если они равны, то осуществляется рекурсивный вызов текущей функции для списка-аргумента без первого и последнего аргументов. Иначе возвращается nil.

\begin{lstlisting}[caption=2-я реализация функции проверки на палиндром]
(defun check_pal (lst) 
	(cond 
		(
			(null lst)
		)
		(
			(reduce 
				#'(lambda (x y) (and x y)) 
				(mapcar #'equal lst (reverse lst))
			)
		)
	)
) 
\end{lstlisting}
\textbf{lst} - входной список.

Осуществляется полный перебор входного списка и его перевернутой с помощью reverse копии. С помощью mapcar сравниваются их элементы, а затем с помощью reduce проверяется, есть ли несовпадающие элементы.

\subsubsection*{Примеры работы:}
\begin{lstlisting}
> (check_pal '(a))
T
> (check_pal nil)
T
> (check_pal '(a b))
NIL
> (check_pal '(a b a))
T
> (check_pal '(a b b a))
T
> (check_pal '((a b) c d c (a b)))
T
\end{lstlisting}

\subsection*{Задание 4}

Напишите функцию swap-first-last, которая переставляет в списке аргументе первый и последний элементы.

\begin{lstlisting}[caption=Функция перестановки первого и последнего элементов]
(defun swap-first-last(lst)
	(cond 
		(
			(null (cdr lst))
			lst
		)
		(
			(nconc 
				(
					mapcon #'(lambda (el) 
						(cond
							(
								(null (cdr el))
								el
							)
						)
					) lst
				)
				(
					mapcon #'(lambda (el) 
						(cond
							(
								(cdr el)
								(cons (car el) nil) 
							)
						)
					) (cdr lst)
				)
				(cons (car lst) nil)
			)
		)
	)
)
\end{lstlisting}
\textbf{lst} - входной список.

С помощью первого mapcon получается список из последнего элемента списка аргумента, с помощью второй mapcon получается список-аргумент без первого и последнего элементов, и создается список из первого элемента списка-аргумента. Затем эти списки объединяются с помощью nconc.

\subsubsection*{Примеры работы:}
\begin{lstlisting}
> (swap-first-last '(1 2 3 4))
(4 2 3 1)
> (swap-first-last nil)
NIL
> (swap-first-last '(1))
(1)
> (swap-first-last '((1 2) 3 4 (5)))
((5) 3 4 (1 2))
\end{lstlisting}

\subsection*{Задание 6}

Напишите две функции, swap-to-left и swap-to-right, которые производят круговую перестановку в списке-аргументе влево и вправо, соответственно на k позиций.

\begin{lstlisting}[caption=Функция круговой перестановки элементов списка влево]
(defun swap-to-left (lst k)
	(cond 
		(
			(<= k 0) 
			lst
		)
		(
			(swap-to-left 
				(nconc
						(cdr lst)
						(cons (car lst) nil)
				)
				(- k 1)
			)
		)
	)
)
\end{lstlisting}
\textbf{lst} - входной список, \textbf{k} - количество позиций, на которое необходимо выполнить перестановку.

Условием выхода из рекурсии является окончание перестановки элементов (второй аргумент - 0) - возвращается первый аргумент.

С помощью nconc объединяется хвост списка-аргумента со списком, указатель на голову которого указывает на голову списка-аргумента, а указатель на хвост - на nil. Затем осуществляется рекурсивный вызов текущей функции для созданного списка и второго аргумента, уменьшенного на 1.
\begin{lstlisting}[caption=Функция круговой перестановки элементов списка вправо]
(defun swap-to-right (lst k)
	(cond 
		(
			(<= k 0) 
			lst
		)
		(
			(swap-to-right 
				(nconc 
					(mapcon #'(lambda (el)
							(cond
								(
									(null (cdr el))
									el
								)
							)
						) lst
					)
					(mapcon #'(lambda (el)
							(cond
								(
									(cdr el)
									(cons (car el) nil)
								)
							)
						) lst
					)
				)
				(- k 1)
			)
		)
	)
)
\end{lstlisting}
\textbf{lst} - входной список, \textbf{k} - количество позиций, на которое необходимо выполнить перестановку.

Условием выхода из рекурсии является окончание перестановки элементов (второй аргумент - 0) - возвращается первый аргумент.

С помощью первого mapcon осуществляется получение списка, указатель на голову которого указывает на последний элемент списка-аргумента, а указатель на хвост - на nil. С помощью второго mapcon осуществляется получение списка-аргумента без последнего элемента. Затем с помощью nconc происходит объединение этих двух списков. Далее осуществляется рекурсивный вызов текущей функции для созданного списка и второго аргумента, уменьшенного на 1.

\subsubsection*{Примеры работы:}
\begin{lstlisting}
> (swap-to-right nil 3)
NIL
> (swap-to-right '(1) 3)
(1)
> (swap-to-right '(1 2 3 4 5) 1)
(5 1 2 3 4)
> (swap-to-right '(1 2 3 4 5) 3)
(3 4 5 1 2)
> (swap-to-right '(1 2 3 4 5) 5)
(1 2 3 4 5)
> (swap-to-right '((1) 2 3 (4 5)) 1)
((4 5) (1) 2 3)
> (swap-to-left '(1 2 3 4 5) 1)
(2 3 4 5 1)
> (swap-to-left '(1 2 3 4 5) 3)
(4 5 1 2 3)
> (swap-to-left '(1 2 3 4 5) 5)
(1 2 3 4 5)
> (swap-to-left '((1) 2 3 (4 5)) 1)
(2 3 (4 5) (1))
\end{lstlisting}

\subsection*{Задание 7}

Напишите функцию, которая умножает на заданное число-аргумент все числа из заданного списка-аргумента, когда
\begin{enumerate}
	\item[а)] все элеметны списка - числа,
	\item[б)] элементы списка - любые объекты.\\
\end{enumerate}

\begin{lstlisting}[caption=Функция умножения для списка из чисел]
(defun mult_num (lst k)
	(mapcar 
		#'(lambda (el) (* el k))
		lst
	)
)
\end{lstlisting}
\textbf{lst} - входной список, \textbf{k} - число, на которое выполняется умножение.

С помощью mapcar осуществляется формирование списка, состоящего из элементов списка-аргумента, умноженных на число-аргумент.

\begin{lstlisting}[caption=Функция умножения для списка из любых объектов]
(defun mult_all (lst k)
	(mapcar #'(lambda (el)
			(cond
				(
					(numberp el)
					(* el k)
				)
				(
					(atom el)
					el
				)
				(
					(mult_all el k)
				)
			)
		) lst
	)
)
\end{lstlisting}
\textbf{lst} - входной список, \textbf{k} - число, на которое выполняется умножение.

С помощью mapcar существляется проход по всему списку и проверка типа каждого элемента: 
\begin{itemize}
	\item если он является числом, то производится умножение на число-аргумент функции и возвращается результат;
	\item если он является атомом, то он возвращается;
	\item если он является списком, то к нему рекурсивно применяется текущая функция.
\end{itemize}

\subsubsection*{Примеры работы:}
\begin{lstlisting}
> (mult_num nil 1)
NIL
> (mult_num '(1) 3)
(3)
> (mult_num '(1 2 3 4 5) 3)
(3 6 9 12 15)
> (mult_all nil 2)
NIL
> (mult_all '(1) 2)
(2)
> (mult_all '(1 2 3) 3)
(3 6 9)
> (mult_all '((1 2) 3 (4 (5))) 3)
((3 6) 9 (12 (15)))
> (mult_all '(1 2 (a)) 3)
(3 6 (A))
\end{lstlisting}

\subsection*{Задание 8}

Напишите функцию, select-between, которая из списка-аргумента, содержащего только числа, выбирает только те, которые расположены между двумя указанными границами-аргументами и возвращает их в виде списка.

\begin{lstlisting}[caption=Функция выборки из списка по границам]
(defun select-between (lst a b)
	(remove-if	#'(lambda (x) 
			(and 
				(or (< x a)(> x b)) 
				(or (> x a)(< x b))
			)
		)lst
	)
)
\end{lstlisting}
\textbf{lst} -  входной список, \textbf{a}, \textbf{b}  -  границы выбора.

С помощью функционала remove-if осуществляется проход по всему списку-аргументу и удаление тех элементов списка, которые не входят в заданные границы-аргументы.

\subsubsection*{Примеры работы:}
\begin{lstlisting}
> (select-between nil 1 2)
NIL
> (select-between '(1) 0 2)
(1)
> (select-between '(1 2 3) 4 5)
NIL
> (select-between '(1 2 3 4 5) 2 4)
(2 3 4)
> (select-between '(1 2 3 4 5) 4 2)
(2 3 4)
\end{lstlisting}
\end{document}