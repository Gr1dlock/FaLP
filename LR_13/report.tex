\documentclass[a4paper,12pt]{article}

\usepackage{cmap}					% поиск в PDF
\usepackage{mathtext} 				% русские буквы в формулах
\usepackage[T2A]{fontenc}			% кодировка
\usepackage[utf8]{inputenc}		% кодировка исходного текста
\usepackage[english,russian]{babel}	% локализация и переносы
\usepackage{indentfirst}
\frenchspacing

\usepackage[
bookmarks=true, colorlinks=true, unicode=true,
urlcolor=blue,linkcolor=black, anchorcolor=black,
citecolor=black, menucolor=black, filecolor=black,
]{hyperref}

\usepackage{caption}
\usepackage{csquotes}
\usepackage{multicol}

\usepackage{amsmath,amsfonts,amssymb,amsthm,mathtools} 
\usepackage{wasysym}
\usepackage{icomma}
\usepackage{wrapfig}

\usepackage{graphicx}

\usepackage{listings} 

\usepackage{geometry}
\geometry{left=2cm}
\geometry{right=1.5cm}
\geometry{top=2cm}
\geometry{bottom=2cm}


\usepackage{array,tabularx,tabulary,booktabs} % Дополнительная работа с таблицами
\usepackage{longtable}  % Длинные таблицы
\usepackage{multirow} % Слияние строк в таблице

\usepackage{soul}

\usepackage{xcolor}
\DeclareCaptionFont{white}{\color{white}}
\DeclareCaptionFormat{listing}{\colorbox{gray}{\parbox{\textwidth}{#1#2#3}}}
\captionsetup[lstlisting]{format=listing,labelfont=white,textfont=white}


\lstset{ 
	language=Prolog,                 % выбор языка для подсветки
	basicstyle=\small\ttfamily,
	numbers=left,               % где поставить нумерацию строк (слева\справа)
	numberstyle=\tiny\color{black},           % размер шрифта для номеров строк
	stepnumber=1,                   % размер шага между двумя номерами строк
	numbersep=5pt,                % как далеко отстоят номера строк от подсвечиваемого кода
	showspaces=false,
	backgroundcolor=\color{white},         
	showstringspaces=false,      % показывать или нет пробелы в строках
	showtabs=false,             % показывать или нет табуляцию в строках
	frame=single,              % рисовать рамку вокруг кода
	rulecolor = \color{black},
	tabsize=2,                 % размер табуляции по умолчанию равен 2 пробелам
	captionpos=t,              % позиция заголовка вверху [t] или внизу [b] 
	breaklines=true,           % автоматически переносить строки (да\нет)
	breakatwhitespace=true, % переносить строки только если есть пробел
	escapeinside={\%*}{*)},
}





\begin{document}
	
\begin{figure}[h!]
	\begin{center}
		{\includegraphics[width = \textwidth]{titul.png}}
	\end{center}
\end{figure}

\vspace*{20mm}

\huge
\begin{center}
	Лабораторная работа №11\\
	Среда Visual Prolog 5.2
\end{center}


\vspace*{45mm}

\large
\begin{flushleft}
	Студент: Луговой Д.М. \\
	Группа: ИУ7-61Б \\
	Преподаватель: Толпинская Н.Б.
\end{flushleft}

\vspace*{55mm}

\large
\begin{center}
	Москва, 2020 г.
\end{center}

\thispagestyle{empty}

\textbf{Цель работы}: получить навыки построения модели предметной области, разработки и оформления программы на Prolog, изучить принципы, логику формирования программы и отдельные шаги выполнения программы на Prolog.\\

\textbf{Задачи работы}: приобрести навыки декларативного описания предметной области с использованием фактов и правил.
Изучить способы использования термов, переменных, фактов и правил в программе на Prolog, принципы  и правила сопоставления и отождествления, порядок унификации.

\section*{Задание}

Составить программу, т.е. модель предметной области – базу знаний, объединив в ней информацию – знания:
\begin{itemize}
	\item «Телефонный справочник»: Фамилия, №тел, Адрес – структура (Город, Улица, №дома, №кв),
	\item «Автомобили»: Фамилия\_владельца, Марка, Цвет, Стоимость, и др.,
	\item «Вкладчики банков»: Фамилия, Банк, счет, сумма, др.
\end{itemize}

Владелец может иметь несколько телефонов, автомобилей, вкладов (Факты).

Используя правила, обеспечить возможность поиска:
\begin{enumerate}
	\item 
	\begin{enumerate}
		\item По № телефона найти: Фамилию, Марку автомобиля, Стоимость автомобиля (может быть несколько),
		\item Используя сформированное в пункте а) правило, по № телефона найти: только Марку автомобиля (автомобилей может быть несколько),
	\end{enumerate}
	\item Используя простой, не составной вопрос: по Фамилии (уникальна в городе, но в разных городах есть однофамильцы) и Городу проживания найти:  Улицу проживания, Банки, в которых есть вклады и №телефона.
\end{enumerate}

Для задания1 и задания2: 
для одного из вариантов ответов, и для а) и для в), описать словесно порядок поиска ответа на вопрос, указав, как выбираются знания, и, при этом, для каждого этапа унификации, выписать подстановку – наибольший общий унификатор, и соответствующие примеры термов.

\newpage

\subsection*{Текст программы}
\begin{lstlisting}[caption=База знаний]
domains 
name, phone, city, street, color, brand, money , bank, account = symbol. 
house, apartment = integer. 
addr = address(city, street, house, apartment). 

predicates 
phonebook(name, phone, addr). 
car(name, brand, color, money). 
depositor(name, bank, account, money). 
find_name_brand_money(phone, name, brand, money). 
find_street_bank_phone(name, city, street, bank, phone). 

clauses
find_name_brand_money(Phone, Name, Brand, Money) :- phonebook(Name, Phone, _), car(Name, Brand, _, Money). 
find_street_bank_phone(Name, City, Street, Bank, Phone) :- phonebook(Name, Phone, address(City, Street, _, _)), depositor(Name, Bank, _, _). 

phonebook("Ivanov", "79836457823", address("Moscow", "Tverskaya", 4, 112)). 
phonebook("Sidorov", "79285920831", address("Tver", "Orlova", 17, 22)). 
phonebook("Ivanov", "79260112361", address("Moscow", "Tverskaya", 4, 112)). 
phonebook("Petrov", "79256239576", address("St-Petersburg", "Leninskaya", 19, 26)). 
phonebook("Sidorov", "79278456344", address("Moscow", "Puskinskaya", 2, 34)). 

car("Petrov", "BMW", "Black", "3500000"). 
car("Ivanov", "Mercedes", "Black", "5000000"). 
car("Sidorov", "BMW", "Red", "2000000"). 
car("Petrov", "Audi", "Blue", "3000000"). 

depositor("Sidorov", "Sberbank", "1238123127", "5000000"). 
depositor("Ivanov", "Tinkoff", "5872874928", "300000"). 
depositor("Sidorov", "VTB", "123123213", "2000000"). 
depositor("Petrov", "Sberbank", "123213213", "3000000"). 

\end{lstlisting}
Предикат \emph{find\_name\_brand\_money(phone, name, brand, money)} обеспечивает возможность поиска имени, марки автомобиля и его стоимости по номеру телефона, предикат \emph{find\_street\_bank\_phone(name, city, street, bank, phone).} позволяет найти улицу проживания, банк и номер телефона по имени и городу.

 \newpage
 
\subsection*{Примеры работы:}

\begin{table}[ht!] 
	\begin{tabularx}{\linewidth}{|>{\centering}X|>{\centering}X|}
		\hline
		goal & Результат \tabularnewline
		\hline
			find\_name\_brand\_money("79260112361"{}, Name, Brand, Money). &
		Name=Ivanov, Brand=Mercedes, Money=5000000 \\
		1 Solution\tabularnewline
		\hline
		find\_name\_brand\_money("79256239576"{}, \_, Brand, \_). &
		Brand=BMW \\
		Brand=Audi \\
		2 Solutions\tabularnewline
		\hline
find\_street\_bank\_phone("Sidorov"{}, "Moscow"{}, Street, Bank, Phone). & Street=Puskinskaya, Bank=Sberbank, Phone=79278456344 \\
Street=Puskinskaya, Bank=VTB, Phone=79278456344 \\
2 Solutions \tabularnewline
		\hline
	\end{tabularx}
\end{table}

Порядок формирования результата для 1-го вопроса:
\begin{table}[ht!] 
	\begin{tabularx}{\linewidth}{|c|>{\centering}X|>{\centering}X|}
		\hline
		№ шага & Сравниваемые термы; результат; подстановка, если есть & Дальнейшие действия: прямой ход или откат (к чему приводит?)\tabularnewline
		\hline
		1 & Сравниваются find\_name\_brand\_money( "79260112361"{}, Name, Brand, Money) и find\_name\_brand\_money(Phone, Name, Brand, Money), подстановка - \{Phone="79260112361"{}\} & Прямой ход \tabularnewline
		\hline
		2 & Сравниваются phonebook(Name, "79260112361"{}, \_) и find\_name\_brand\_money(Phone, Name, Brand, Money), они имеют разные функторы & Термы не унифицируемы, переход к следующему предложению \tabularnewline
		\hline
		3 & Сравниваются phonebook(Name, "79260112361"{}, \_) и find\_street\_bank\_phone(Name, City, Street, Bank, Phone), они имеют разные функторы & Термы не унифицируемы, переход к следующему предложению \tabularnewline
		\hline
		4 & Сравниваются phonebook(Name, "79260112361"{}, \_) и phonebook("Ivanov"{}, "79836457823"{}, address("Moscow"{}, "Tverskaya"{}, 4, 112)), термы "79260112361"{} и "79836457823"{} не унифицируемы & Термы не унифицируемы, переход к следующему предложению \tabularnewline
		\hline
		5 & Сравниваются phonebook(Name, "79260112361"{}, \_) и phonebook("Sidorov"{}, "79285920831"{}, address("Tver"{}, "Orlova"{}, 17, 22)), термы "79260112361" и "79285920831" не унифицируемы & Термы не унифицируемы, переход к следующему предложению\tabularnewline
		\hline
	\end{tabularx}
\end{table}
\newpage
\begin{table}[ht!] 
	\begin{tabularx}{\linewidth}{|c|>{\centering}X|>{\centering}X|}
		\hline
		6 & Сравниваются phonebook(Name, "79260112361"{}, \_) и phonebook("Ivanov"{}, "79260112361"{}, address("Moscow"{}, "Tverskaya"{}, 4, 112)), подстановка - \{Name="Ivanov"{}, Phone="79260112361"{}, \_=address("Moscow"{}, "Tverskaya"{}, 4, 112)\} & Прямой ход, занесение Name="Ivanov" в результирующую ячейку \tabularnewline
		\hline
		7 & Сравниваются car("Ivanov"{}, Brand, \_, Money) и find\_name\_brand\_money(Phone, Name, Brand, Money), они имеют разные функторы & Термы не унифицируемы, переход к следующему предложению \tabularnewline
		\hline
		8 & Сравниваются car("Ivanov"{}, Brand, \_, Money) и find\_street\_bank\_phone(Name, City, Street, Bank, Phone), они имеют разные функторы & Термы не унифицируемы, переход к следующему предложению \tabularnewline
		\hline
		9 & Сравниваются car("Ivanov"{}, Brand, \_, Money) и phonebook("Ivanov"{}, "79836457823"{}, address("Moscow"{}, "Tverskaya"{}, 4, 112)), они имеют разные функторы & Термы не унифицируемы, переход к следующему предложению \tabularnewline
		\hline
		10 & Сравниваются car("Ivanov"{}, Brand, \_, Money) и phonebook("Sidorov"{}, "79285920831"{}, address("Tver"{}, "Orlova"{}, 17, 22)), они имеют разные функторы & Термы не унифицируемы, переход к следующему предложению\tabularnewline
		\hline
		11 & Сравниваются car("Ivanov"{}, Brand, \_, Money) и phonebook("Ivanov"{}, "79260112361"{}, address("Moscow"{}, "Tverskaya"{}, 4, 112)), они имеют разные функторы & Термы не унифицируемы, переход к следующему предложению\tabularnewline
		\hline
		12 & Сравниваются car("Ivanov"{}, Brand, \_, Money) и  phonebook("Petrov"{}, "79256239576"{}, address("St-Petersburg"{}, "Leninskaya"{}, 19, 26)), они имеют разные функторы & Термы не унифицируемы, переход к следующему предложению\tabularnewline
		\hline
		13 & Сравниваются car("Ivanov"{}, Brand, \_, Money) и  phonebook("Sidorov"{}, "79278456344"{}, address("Moscow"{}, "Puskinskaya"{}, 2, 34)), они имеют разные функторы & Термы не унифицируемы, переход к следующему предложению\tabularnewline
		\hline
		14 & Сравниваются car("Ivanov"{}, Brand, \_, Money) и car("Petrov"{}, "BMW"{}, "Black"{}, "3500000"), термы "Ivanov" и "Petrov" не унифицируемы & Термы не унифицируемы, переход к следующему предложению \tabularnewline
		\hline
		15 & Сравниваются car("Ivanov"{}, Brand, \_, Money) и car("Ivanov"{}, "Mercedes"{}, "Black"{}, "5000000"), подстановка - \{Brand="Mercedes"{}, \_="Black"{}, Money="5000000"\} & Прямой ход, занесение Brand="Mercedes"{}, Money="5000000" в результирующую ячейку \tabularnewline
		\hline
	\end{tabularx}
\end{table}
\newpage
\begin{table}[ht!] 
	\begin{tabularx}{\linewidth}{|c|>{\centering}X|>{\centering}X|}
		\hline
		16 & Результат: подстановка \{Name="Ivanov"{}, Brand="Mercedes"{}, Money="5000000"\} &  \tabularnewline
		\hline
	\end{tabularx}
\end{table}	

Порядок формирования результата для 2-го вопроса:
\begin{table}[!ht]
	\begin{tabularx}{\linewidth}{|c|>{\centering}X|>{\centering}X|}
		\hline
		№ шага & Сравниваемые термы; результат; подстановка, если есть & Дальнейшие действия: прямой ход или откат (к чему приводит?)\tabularnewline
		\hline
		1 & Сравниваются find\_name\_brand\_money( "79256239576"{}, \_, Brand, \_) и find\_name\_brand\_money(Phone, Name, Brand, Money), подстановка - \{Phone="79256239576"\} &  Прямой ход  \tabularnewline
		\hline
		2 & Сравниваются phonebook(Name, "79256239576"{}, \_) и find\_name\_brand\_money(Phone, Name, Brand, Money), они имеют разные функторы & Термы не унифицируемы, переход к следующему предложению \tabularnewline
		\hline
		3 & Сравниваются phonebook(Name, "79256239576"{}, \_) и find\_street\_bank\_phone(Name, City, Street, Bank, Phone), они имеют разные функторы & Термы не унифицируемы, переход к следующему предложению \tabularnewline
		\hline
		4 & Сравниваются phonebook(Name, "79256239576"{}, \_) и phonebook("Ivanov"{}, "79836457823"{}, address("Moscow"{}, "Tverskaya"{}, 4, 112)), термы "79256239576" и "79836457823" не унифицируемы & Термы не унифицируемы, переход к следующему предложению \tabularnewline
		\hline
		5 & Сравниваются phonebook(Name, "79256239576"{}, \_) и phonebook("Sidorov"{}, "79285920831"{}, address("Tver"{}, "Orlova"{}, 17, 22)), термы "79256239576" и "79285920831" не унифицируемы & Термы не унифицируемы, переход к следующему предложению \tabularnewline
		\hline
		6 & Сравниваются phonebook(Name, "79256239576"{}, \_) и phonebook("Ivanov"{}, "79260112361"{}, address("Moscow"{}, "Tverskaya"{}, 4, 112)), термы "79256239576" и "79260112361" не унифицируемы & Термы не унифицируемы, переход к следующему предложению \tabularnewline
		\hline
		7 & Сравниваются phonebook(Name, "79256239576"{}, \_) и phonebook("Petrov"{}, "79256239576"{}, address("St-Petersburg"{}, "Leninskaya{}", 19, 26)), подстановка - \{Name="Petrov"{}, \_=address("St-Petersburg"{}, "Leninskaya{}", 19, 26)\} & Прямой ход \tabularnewline
		\hline
	\end{tabularx}
\end{table}
\newpage
\begin{table}[ht!] 
	\begin{tabularx}{\linewidth}{|c|>{\centering}X|>{\centering}X|}
		\hline
		8 & Сравниваются car("Petrov"{}, Brand, \_, Money) и find\_name\_brand\_money(Phone, Name, Brand, Money), они имеют разные функторы & Термы не унифицируемы, переход к следующему предложению \tabularnewline
		\hline
		9 & Сравниваются car("Petrov"{}, Brand, \_, Money) и find\_street\_bank\_phone(Name, City, Street, Bank, Phone), они имеют разные функторы & Термы не унифицируемы, переход к следующему предложению \tabularnewline
		\hline
		10 & Сравниваются car("Petrov"{}, Brand, \_, Money) и phonebook("Ivanov"{}, "79836457823"{}, address("Moscow"{}, "Tverskaya"{}, 4, 112)), они имеют разные функторы & Термы не унифицируемы, переход к следующему предложению \tabularnewline
		\hline
		11 & Сравниваются car("Petrov"{}, Brand, \_, Money) и phonebook("Sidorov"{}, "79285920831"{}, address("Tver"{}, "Orlova"{}, 17, 22)), они имеют разные функторы & Термы не унифицируемы, переход к следующему предложению\tabularnewline
		\hline
		12 & Сравниваются car("Petrov"{}, Brand, \_, Money) и phonebook("Ivanov"{}, "79260112361"{}, address("Moscow"{}, "Tverskaya"{}, 4, 112)), они имеют разные функторы & Термы не унифицируемы, переход к следующему предложению\tabularnewline
		\hline
		13 & Сравниваются car("Petrov"{}, Brand, \_, Money) и  phonebook("Petrov"{}, "79256239576"{}, address("St-Petersburg"{}, "Leninskaya"{}, 19, 26)), они имеют разные функторы & Термы не унифицируемы, переход к следующему предложению\tabularnewline
		\hline
		14 & Сравниваются car("Petrov"{}, Brand, \_, Money) и  phonebook("Sidorov"{}, "79278456344"{}, address("Moscow"{}, "Puskinskaya"{}, 2, 34)), они имеют разные функторы & Термы не унифицируемы, переход к следующему предложению\tabularnewline
		\hline
		15 & Сравниваются car("Petrov"{}, Brand, \_, Money) и car("Petrov"{}, "BMW"{}, "Black"{}, "3500000"), подстановка - \{Brand="BMW"{}, \_="Black"{}, Money="3500000"\} & Прямой ход, занесение Brand="BMW" в результирующую ячейку \tabularnewline
		\hline
		16 & Результат: подстановка \{Brand="BMW"\} &  \tabularnewline
		\hline
\end{tabularx}
\end{table}
		
\newpage
Порядок формирования результата для 3-го вопроса:
\begin{table}[ht!] 
	\begin{tabularx}{\linewidth}{|c|>{\centering}X|>{\centering}X|}
		\hline
		№ шага & Сравниваемые термы; результат; подстановка, если есть & Дальнейшие действия: прямой ход или откат (к чему приводит?)\tabularnewline
		\hline
		1 & Сравниваются find\_street\_bank\_phone("Sidorov"{}, "Moscow"{}, Street, Bank, Phone) и find\_name\_brand\_money(Phone, Name, Brand, Money), они имеют разные функторы & Термы не унифицируемы, переход к следующему предложению \tabularnewline
		\hline
		2 & Сравниваются find\_street\_bank\_phone("Sidorov"{}, "Moscow"{}, Street, Bank, Phone) и find\_street\_bank\_phone(Name, City, Street, Bank, Phone), подстановка - \{Name="Sidorov"{}, City="Moscow\} & Прямой ход \tabularnewline
		\hline
		3 & Сравниваются phonebook("Sidorov"{}, Phone, address("Moscow"{}, Street, \_, \_)) и find\_name\_brand\_money(Phone, Name, Brand, Money), они имеют разные функторы & Термы не унифицируемы, переход к следующему предложению \tabularnewline
		\hline
		4 & Сравниваются phonebook("Sidorov"{}, Phone, address("Moscow"{}, Street, \_, \_)) и find\_street\_bank\_phone(Name, City, Street, Bank, Phone), они имеют разные функторы & Термы не унифицируемы, переход к следующему предложению \tabularnewline
		\hline
		5 & Сравниваются phonebook("Sidorov"{}, Phone, address("Moscow"{}, Street, \_, \_)) и phonebook("Ivanov"{}, "79836457823"{}, address("Moscow"{}, "Tverskaya"{}, 4, 112)), термы "Sidorov" и "Ivanov" не унифицируемы & Термы не унифицируемы, переход к следующему предложению \tabularnewline
		\hline
		6 & Сравниваются phonebook("Sidorov"{}, Phone, address("Moscow"{}, Street, \_, \_)) и phonebook("Sidorov"{}, "79285920831"{}, address("Tver"{}, "Orlova"{}, 17, 22)), термы address("Moscow"{}, Street, \_, \_) и address("Tver"{}, "Orlova"{}, 17, 22) не унифицируемы, так как термы "Moscow"{} и "Tver" не унифицируемы & Термы не унифицируемы, переход к следующему предложению\tabularnewline
		\hline
		7 & Сравниваются phonebook("Sidorov"{}, Phone, address("Moscow"{}, Street, \_, \_)) и phonebook("Ivanov"{}, "79260112361"{}, address("Moscow"{}, "Tverskaya"{}, 4, 112)), термы "Sidorov"{} и "Ivanov"{} не унифицируемы & Термы не унифицируемы, переход к следующему предложению\tabularnewline
		\hline
	\end{tabularx}
\end{table}
\newpage
\begin{table}[ht!] 
	\begin{tabularx}{\linewidth}{|c|>{\centering}X|>{\centering}X|}
		\hline
		8 & Сравниваются phonebook("Sidorov"{}, Phone, address("Moscow"{}, Street, \_, \_)) и  phonebook("Petrov"{}, "79256239576"{}, address("St-Petersburg"{}, "Leninskaya"{}, 19, 26)), термы "Sidorov"{} и "Petrov"{} не унифицируемы & Термы не унифицируемы, откат, переход к следующему предложению\tabularnewline
		\hline
		9 & Сравниваются phonebook("Sidorov"{}, Phone, address("Moscow"{}, Street, \_, \_)) и  phonebook("Sidorov"{}, "79278456344"{}, address("Moscow"{}, "Puskinskaya"{}, 2, 34)), подстановка \{Phone="79278456344"{}, Street="Puskinskaya"{}, \_=2, \_=34\} & Прямой ход, занесение Phone="79278456344"{}, Street="Puskinskaya"{} в результирующую ячейку\tabularnewline
		\hline
		10 & Сравниваются depositor("Sidorov"{}, Bank, \_, \_) и find\_name\_brand\_money(Phone, Name, Brand, Money), они имеют разные функторы & Термы не унифицируемы, переход к следующему предложению \tabularnewline
		\hline
		11 & Сравниваются depositor("Sidorov"{}, Bank, \_, \_) и find\_street\_bank\_phone(Name, City, Street, Bank, Phone), они имеют разные функторы & Термы не унифицируемы, переход к следующему предложению \tabularnewline
		\hline
		12 & Сравниваются depositor("Sidorov"{}, Bank, \_, \_) и phonebook("Ivanov"{}, "79836457823"{}, address("Moscow"{}, "Tverskaya"{}, 4, 112)), они имеют разные функторы & Термы не унифицируемы, переход к следующему предложению \tabularnewline
		\hline
		13 & Сравниваются depositor("Sidorov"{}, Bank, \_, \_) и phonebook("Sidorov"{}, "79285920831"{}, address("Tver"{}, "Orlova"{}, 17, 22)), они имеют разные функторы & Термы не унифицируемы, переход к следующему предложению \tabularnewline
		\hline
		14 & Сравниваются depositor("Sidorov"{}, Bank, \_, \_) и phonebook("Ivanov"{}, "79260112361"{}, address("Moscow"{}, "Tverskaya"{}, 4, 112)), они имеют разные функторы & Термы не унифицируемы, переход к следующему предложению \tabularnewline
		\hline
		15 & Сравниваются depositor("Sidorov"{}, Bank, \_, \_) и phonebook("Petrov"{}, "79256239576"{}, address("St-Petersburg"{}, "Leninskaya"{}, 19, 26)), они имеют разные функторы & Термы не унифицируемы, переход к следующему предложению \tabularnewline
		\hline
		16 & Сравниваются depositor("Sidorov"{}, Bank, \_, \_) и phonebook("Sidorov"{}, "79278456344"{}, address("Moscow"{}, "Puskinskaya"{}, 2, 34)), они имеют разные функторы & Термы не унифицируемы, переход к следующему предложению \tabularnewline
		\hline
		17 & Сравниваются depositor("Sidorov"{}, Bank, \_, \_) и car("Petrov"{}, "BMW"{}, "Black"{}, "3500000"), они имеют разные функторы & Термы не унифицируемы, переход к следующему предложению \tabularnewline
		\hline
	\end{tabularx}
\end{table}
\newpage
\begin{table}[ht!] 
	\begin{tabularx}{\linewidth}{|c|>{\centering}X|>{\centering}X|}
		\hline
		18 & Сравниваются depositor("Sidorov"{}, Bank, \_, \_) и car("Ivanov"{}, "Mercedes"{}, "Black"{}, "5000000"), они имеют разные функторы & Термы не унифицируемы, переход к следующему предложению \tabularnewline
		\hline
		19 & Сравниваются depositor("Sidorov"{}, Bank, \_, \_) и car("Sidorov"{}, "BMW"{}, "Red"{}, "2000000"), они имеют разные функторы & Термы не унифицируемы, переход к следующему предложению \tabularnewline
		\hline
		20 & Сравниваются depositor("Sidorov"{}, Bank, \_, \_) и car("Petrov"{}, "Audi"{}, "Blue"{}, "3000000"), они имеют разные функторы & Термы не унифицируемы, переход к следующему предложению \tabularnewline
		\hline
		21 & Сравниваются depositor("Sidorov"{}, Bank, \_, \_) и depositor("Sidorov"{}, "Sberbank"{}, "1238123127"{}, "5000000"), подстановка \{Bank="Sberbank"{}, \_="1238123127"{}, \_="5000000"\}& Прямой ход, занесение Bank="Sberbank" в результирующую ячейку\tabularnewline
		\hline
		22 & Результат: подстановка \{Phone="79278456344"{}, Street="Puskinskaya"{}, Bank="Sberbank"\} & \tabularnewline
		\hline
	\end{tabularx}
\end{table}	

\section*{Теоретические вопросы}

\subsection*{Что такое терм?}
Терм - основной элемент языка Prolog. Терм – это:
\begin{enumerate}
	\item Константа: 
	\begin{itemize}
		\item Число (целое, вещественное),
		\item Символьный атом (комбинация символов латинского алфавита, цифр и символа подчеркивания, начинающаяся со строчной буквы),
		\item Строка: последовательность символов, заключенных в кавычки.
	\end{itemize}
	\item Переменная:
	\begin{itemize}
		\item Именованная – обозначается комбинацией символов латинского алфавита, цифр и символа подчеркивания, начинающейся с прописной буквы или символа подчеркивания,
		\item Анонимная  - обозначается символом подчеркивания
	\end{itemize}
	\item Составной терм:
	Это средство организации группы отдельных элементов знаний в единый  объект,  синтаксически представляется: f(t1, t2, …,tm), где f -  функтор (отношение между объектами), t1, t2, …,tm – термы, в том  числе  и составные.
\end{enumerate}
\subsection*{Что такое предикат в матлогике (математике)?}
Предикат в математической логике - это утверждение, высказанное о субъекте. Предикат является функцией со значениями {0, 1} (истина/ложь), определенной на некотором множестве параметров. Предикат называю n-арным, если он определен на n-ой декартовой степени множества М.
\subsection*{Что описывает предикат в Prolog?}
Предикат в Prolog описывает отношение между аргументами процедуры. Процедурой в Prolog является совокупность всех правил, описывающих определенное отношение.
\subsection*{Назовите виды предложений в программе и приведите примеры таких предложений из Вашей программы. Какие предложения являются основными, а какие – не основными?  Каковы: синтаксис и семантика (формальный смысл) этих предложений (основных и неосновных)?}
В Prolog есть два типа предложений: правила и факты. Правило имеет вид: A :- B1,... , Bn. 
A называется заголовком правила, а B1,..., Bn – телом правила. Заголовок содержит некоторое знание, а тело - условие истинности этого знания. Факт является частным случаем правила - в нем отсутствует тело.

Пример факта из программы: \emph{car("Petrov"{}, "BMW"{}, "Black"{}, "3500000"). }

Пример правила из программы: \emph{find\_name\_brand\_money(Phone, Name, Brand, Money) :- phonebook(Name, Phone, \_), car(Name, Brand, \_, Money). }

Основными называются предложения, не содержащие переменных. Предложения, содержащие переменные называются неосновными. 

Синтаксис предложения: заголовок(составной терм) :- тело(один или последовательность термов). Предложения используются для формирования базы знаний о некоторой предметной области.
\subsection*{Каковы назначение, виды и особенности использования переменных в программе на Prolog? Какое предложение БЗ сформулировано в более общей – абстрактной форме: содержащее или не содержащее переменных?}
Переменные предназначены для обозначения некоторого неизвестного объекта предметной области. Переменные бывают именованными и анонимными. Именованные переменные уникальны в рамках предложения, а анонимная переменная – любая уникальна. В разных предложениях может использоваться одно имя переменной для обозначения разных объектов.

В ходе выполнения программы выполняется связывание переменных с различными объектами, этот процесс называется конкретизацией. Это относится только к именованным переменным. Анонимные переменные не могут быть связаны со значением.

В более общей форме сформулировано предложение, содержащее переменные, так как заранее неизвестно, каким объектом будет конкретизирована переменная.
\subsection*{Что такое подстановка?}
Пусть дан терм: $А(X_1, X_2,  \dots ,X_n)$.
Подстановка - множество пар, вида: \\ $\{X _ i = t _ i\}$, где $X_i$ –   переменная, а $t_i$ –  терм.
\subsection*{Что такое пример терма? Как и когда строится? Как Вы думаете, система строит и хранит примеры?}
Пусть $\Theta =  \{X_1 = t_1, X_2= t_2, \dots , X_n = t_n \}$   –   подстановка, $A$ - терм. Результат применения подстановки к терму обозначается $A\Theta$.

Примером терма $A$ называется терм $B$, если существует подстановка $\Theta$ такая, что $B = A\Theta$.

Примеры термов строятся в ходе логического вывода. Для построения примера терма его переменные конкретизируются.

\end{document}