\usepackage{cmap}					% поиск в PDF
\usepackage{mathtext} 				% русские буквы в формулах
\usepackage[T2A]{fontenc}			% кодировка
\usepackage[utf8]{inputenc}		% кодировка исходного текста
\usepackage[english,russian]{babel}	% локализация и переносы
\usepackage{indentfirst}
\frenchspacing

\usepackage[
bookmarks=true, colorlinks=true, unicode=true,
urlcolor=blue,linkcolor=black, anchorcolor=black,
citecolor=black, menucolor=black, filecolor=black,
]{hyperref}

\usepackage{caption}
\usepackage{csquotes}
\usepackage{multicol}

\usepackage{amsmath,amsfonts,amssymb,amsthm,mathtools} 
\usepackage{wasysym}
\usepackage{icomma}
\usepackage{wrapfig}

\usepackage{graphicx}

\usepackage{listings} 

\usepackage{geometry}
\geometry{left=2cm}
\geometry{right=1.5cm}
\geometry{top=2cm}
\geometry{bottom=2cm}


\usepackage{array,tabularx,tabulary,booktabs} % Дополнительная работа с таблицами
\usepackage{longtable}  % Длинные таблицы
\usepackage{multirow} % Слияние строк в таблице

\usepackage{soul}

\usepackage{xcolor}
\DeclareCaptionFont{white}{\color{white}}
\DeclareCaptionFormat{listing}{\colorbox{gray}{\parbox{\textwidth}{#1#2#3}}}
\captionsetup[lstlisting]{format=listing,labelfont=white,textfont=white}


\lstset{ 
	language=Prolog,                 % выбор языка для подсветки
	basicstyle=\small\ttfamily,
	numbers=left,               % где поставить нумерацию строк (слева\справа)
	numberstyle=\tiny\color{black},           % размер шрифта для номеров строк
	stepnumber=1,                   % размер шага между двумя номерами строк
	numbersep=5pt,                % как далеко отстоят номера строк от подсвечиваемого кода
	showspaces=false,
	backgroundcolor=\color{white},         
	showstringspaces=false,      % показывать или нет пробелы в строках
	showtabs=false,             % показывать или нет табуляцию в строках
	frame=single,              % рисовать рамку вокруг кода
	rulecolor = \color{black},
	tabsize=2,                 % размер табуляции по умолчанию равен 2 пробелам
	captionpos=t,              % позиция заголовка вверху [t] или внизу [b] 
	breaklines=true,           % автоматически переносить строки (да\нет)
	breakatwhitespace=true, % переносить строки только если есть пробел
	escapeinside={\%*}{*)},
}





\begin{document}
	
\begin{figure}[h!]
	\begin{center}
		{\includegraphics[width = \textwidth]{titul.png}}
	\end{center}
\end{figure}

\vspace*{20mm}

\huge
\begin{center}
	Лабораторная работа №4
\end{center}


\vspace*{50mm}

\large
\begin{flushleft}
	Студент: Луговой Д.М. \\
	Группа: ИУ7-61Б \\
	Преподаватель: Толпинская Н.Б.
\end{flushleft}

\vspace*{60mm}

\large
\begin{center}
	Москва, 2020 г.
\end{center}

\thispagestyle{empty}

\newpage
\vspace*{10mm}
\textbf{Цель работы}: приобрести навыки создания и использования функций пользователя в Lisp.\\

\textbf{Задачи работы}: изучить способы создания и использования именованных и неименованных функций пользователя для обработки списков.

\begin{enumerate}
\item \textbf{Синтаксическое оформление LISP}
\item \textbf{Трактовка элементов списка}
\item \textbf{Порядок реализации}
\end{enumerate}

{\LARGE Задание №1}\\

Написать программу, которая переводит температуру в системе Фаренгейта в температуру по Цельсию.

(defun f-to-c (f) (* (/ 5.0 9.0) (- f 32))

Как бы назывался роман Р.Брэдбери "+451 по Фаренгейту" в системе по Цельсию?

(f-to-c 451) -> 232.77779 ("+232.77779 по Цельсию")\\

{\LARGE Задание №2}\\

Что получится при вычисления каждого из выражений?
\begin{enumerate}
\item (list 'cons t NIL)\\
Результат: (CONS T NIL)
\item (eval (eval (list 'cons t NIL)))\\
Результат: The function COMMON-LISP:T is undefined.
\item (apply \#'cons '(t NIL))\\
Результат: (T)
\item (list 'eval NIL)\\
Результат: (EVAL NIL)
\item (eval (list 'cons t NIL))\\
Результат: (T)
\item (eval NIL)\\
Результат: NIL
\item (eval (list 'eval NIL))\\
Результат: NIL\\
\end{enumerate}

{\LARGE Задание №3}\\

Написать функцию, вычисляющую катет по заданной гипотенузе и другому катету прямоугольного треугольника, и составить диаграмму ее вычисления.

(defun cathet (hyp cat) (sqrt (- (* hyp hyp) (* cat cat))))

Пример: (cathet 5 4) -> 3.0\\

{\LARGE Задание №4}\\

Написать функцию, вычисляющую площадь трапеции по ее основаниям и высоте, и составить диаграмму ее вычисления.

(defun square (a b h) (* (+ a b) h 0.5))

Пример: (square 1 3 4) -> 8.0

\end{document}